\documentclass[10pt,oneside,sloak,a4paper]{article}
\usepackage[utf8]{inputenc}
\usepackage[slovak]{babel}
\usepackage{amsmath}
\usepackage{amsfonts}
\usepackage{amssymb}

\title{5G a umelá inteligencia vo vzdelávaný\thanks{Semestrálny projekt v predmete Metódy inžinierskej práce, ak. rok 2020/21, vedenie:}} 

\author{Juraj Ševčík\\[2pt]
	{\small Slovenská technická univerzita v Bratislave}\\
	{\small Fakulta informatiky a informačných technológií}\\
	{\small \texttt{xsevcik@stuba.sk}}
}

\date{\small 20. oktober 2020}


\begin{document}

\maketitle

\begin{abstract}


\end{abstract}
V súčasnosti  keď je elektronické vzdelávanie dôležitejšie ako kedykoľvek predtým nenadišli vhodný čas zamerať sa na to kam až sa dá z dnešnou technológiou posunúť a ako ju najvhodnejšie implementovať. Z postupným nástupom 5G a znižovania latencie medzi používateľom a serverom na 1-3 ms sa otvára celí nový smer implementácie umelej inteligencia z možnosťou prispôsobiť učebné postupy a metódy v reálnom čase na základe špecifických potrieb a pokrokov študenta.  5G je taktiež ideálnym prostriedkom pre študentov so špeciálnymi potrebami ci pre učiteľov aby dostali lepšiu a rýchlejšiu odozvou od študentov-
Umelá inteligencia je vďaka už spomenutej nízkej latencii schopná nie len prispôsobovať študijný plán ale ho aj v reálnom čase analyzovať  a vyhodnocovať v pre nás zatiaľ nemysliteľných smeroch. Preto je potrebné sa sústrediť na tvorbu vzdelávacích platforiem ktoré integrujú 5G a umelú inteligenciu.

\newpage
\section{Úvod}
V súčasnosti  keď je elektronické vzdelávanie dôležitejšie ako kedykoľvek predtým nadišiel vhodný čas zamerať sa na to kam až sa dá z dnešnou technológiou posunúť a ako ju najvhodnejšie implementovať. Z postupným nástupom 5G a znižovania latencie medzi používateľom a serverom na 1-3 ms sa otvára celí nový smer implementácie umelej inteligencia z možnosťou prispôsobiť učebné postupy a metódy v reálnom čase na základe špecifických potrieb a pokrokov študenta. 5G je taktiež ideálnym prostriedkom pre študentov so špeciálnymi potrebami či pre učiteľov aby dostali lepšiu a rýchlejšiu odozvou od študentov a ohladom ich pokroku. 

5G je môžemem adaptovať aj na špecifické odbory kde majú velké využitie ako napríklad pre študentov vysoých škôl ktorý vdaka vačšiemu množstvu prenášaných dát budú mať možnosť vyrtuálne preštudovať pamiatok bežne neprístupních návštevníkom čím by umožnily študentom vyrtuálne navštiviť katakombi pod piramídami v Gýze čí Lascauckú jaskinu vo Francúzku. Tiež by mohli študenťi chémie ktorý môžu vykonávať experimenty vo vyrtálnom priestore ktorý by bol napojený na hlavný počítač ktorý by zabespečoval výkon ktorý nedokážu poskitnúť "menšie" zariadenia.


%Umelá inteligencia je vďaka už spomenutej nízkej latencii schopná nie len prispôsobovať študijný plán ale ho aj v reálnom čase analyzovať  a vyhodnocovať v pre nás zatiaľ nemysliteľných smeroch. Preto je potrebné sa sústrediť na tvorbu vzdelávacích platforiem ktoré integrujú 5G a umelú inteligenciu.

\subsection{čo je to evzdelavanie}
Podla oxfordskeho slovníka e-vzdelavanie je systém vzdelávania ktorý využíva elektronické médiá zvičajne cez internne. 

Pod evzdelávaním teda rozumieme akékolvek zvdélávanie pomocou interneru či íných elektronických zariadení od mobilných telefónov či tabletov až po sústavy na vyrtuálnu realitu či najrôznejšie softverové časti ako webstaránky či vyučovacie softvery. 
\subsection{Úvod do 5G}
Nová efektivna gtechnologia podporujúca viac zariadený vačšov rýchlostov čiže mlôýeme používať viac zariadený na opercie z vysov spotrebov dát na a mensov latenciou na menšom mieste, okrem toho to otvorý cele nove spektrum technoloogii ako autonomne autá smart city či iné ....
\subsection{uvod do AI}
John McCarthy a Marvin Minsky popísaly tak že: "Umelá inteligencia je akakkolvek aktivita vzkonanaá strojom o ktorej sa dá povedať že keby su vykonával človek musel by použit inteligenciu." toto môže viesť k chybnej interpretacii ....(doplniť vseobecnu definiciu AI)


\section{Možnosti využitia} 
Mnohé programy implementuju AI, a nachádza si miesto od smartfónov cez autonomne autá, inteligentne domácnosti ......a v neposlednom rade aj v školstve zatial prevažne na univerzitách ale postupne úrechádza aj do nižžších ročnikov.  

\subsection{Vyučovacie aplikacie}

\subparagraph{title}{Vyučovacie aplikacie}
vzdelávaci program Texas...stredne a vysoké šloly, vytvára pre študentov prisposobený obsah a odosiela učitelovi pokroky studentov taktiež upravuje obsah na zaklade výsledkovštudentov...



\subparagraph{Blippar}{Vyučovacie aplikacie}
rozšírena realita  (využitím umelej inteligencie na spravne umiestnenie objektov) obohacuje biologiu, geografiu ,fyziku pre dedi nyžších rčníkov....Londín()

\subsection{ine študijne aplikacie}
Dalšie aplikacie na pomoc študentom ako napríklad Nuance, na prepisovanie dovoého textu na poznamky alebo ine na prepisanie textu z tofografie na textový súbor ....matematicke aplikacie ...


\section{Nevýhody}

\subsection{v čom AI zliháva }

\section{psychologický aspekt evzdelavanie }
>nižšia socializácia 
>"neosobnejší" pristup ale možnost adaptovat študijni plan 
>vôčšia pozornost pre jednotlivých študentov ....

\section{ako e vzdelavanie ovplvnilo školstvo}
- dalo učitelom viac času na jednotlivých studentov 
-nove možnosti vzdelavania 

\section{kedy a ako môžeme očakavat 5G a AI v triedach}


\end{document}

\bibliography{bibliografia_dokumentu.txt}
