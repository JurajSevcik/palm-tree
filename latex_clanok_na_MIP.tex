\documentclass[12pt,sloak,a4paper]{article}
\usepackage[utf8]{inputenc}
\usepackage[slovak]{babel}
\usepackage{amsmath}
\usepackage{amsfonts}
\usepackage{amssymb}

\title{5G a umelá inteligencia vo vzdelávaný\thanks{Semestrálny projekt v predmete Metódy inžinierskej práce, ak. rok 2020/21, vedenie:}} 

\author{Juraj Ševčík\\[2pt]
	{\small Slovenská technická univerzita v Bratislave}\\
	{\small Fakulta informatiky a informačných technológií}\\
	{\small \texttt{xsevcik@stuba.sk}}
}

\date{\small 20. oktober 2020}


\begin{document}

\maketitle

\begin{abstract}


\end{abstract}
\newpage
\section{Úvod}
V súčasnosti  keď je elektronické vzdelávanie dôležitejšie ako kedykoľvek predtým nadišiel vhodný čas zamerať sa na to kam až sa dá z dnešnou technológiou posunúť a ako ju najvhodnejšie implementovať. Z postupným nástupom 5G a znižovania latencie medzi používateľom a serverom na 1-3 ms sa otvára celí nový smer implementácie umelej inteligencia z možnosťou prispôsobiť učebné postupy a metódy v reálnom čase na základe špecifických potrieb a pokrokov študenta. 5G je taktiež ideálnym prostriedkom pre študentov so špeciálnymi potrebami či pre učiteľov aby dostali lepšiu a rýchlejšiu odozvou od študentov a ohladom ich pokroku. 

5G je môžemem adaptovať aj na špecifické odbory kde majú velké využitie ako napríklad pre študentov vysoých škôl ktorý vdaka vačšiemu množstvu prenášaných dát budú mať možnosť vyrtuálne preštudovať pamiatok bežne neprístupních návštevníkom čím by umožnily študentom vyrtuálne navštiviť katakombi pod piramídami v Gýze čí Lascauckú jaskinu vo Francúzku. Tiež by mohli študenťi chémie ktorý môžu vykonávať experimenty vo vyrtálnom priestore ktorý by bol napojený na hlavný počítač ktorý by zabespečoval výkon ktorý nedokážu poskitnúť "menšie" zariadenia.


%Umelá inteligencia je vďaka už spomenutej nízkej latencii schopná nie len prispôsobovať študijný plán ale ho aj v reálnom čase analyzovať  a vyhodnocovať v pre nás zatiaľ nemysliteľných smeroch. Preto je potrebné sa sústrediť na tvorbu vzdelávacích platforiem ktoré integrujú 5G a umelú inteligenciu.

\subsection{čo je to evzdelavanie}
Podla oxfordskeho slovníka e-vzdelavanie je systém vzdelávania ktorý využíva elektronické médiá zvičajne cez internne. 

Pod evzdelávaním teda rozumieme akékolvek zvdélávanie pomocou interneru či íných elektronických zariadení od mobilných telefónov či tabletov až po sústavy na vyrtuálnu realitu či najrôznejšie softverové časti ako webstaránky či vyučovacie softvery. 
\subsection{Úvod do 5G}



\section{Možnosti využitia} 

\subsection{obmedzenia}



\end{document}